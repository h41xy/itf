\section{Vorbereitung}
Nach dem entpacken des Images stellten wir die in der Vorlesung erwähnte Abweichung der MD5 Summe fest.

Die erwartete Summe war
\begin{center}
4afc088a94dd6c36e750b7462e737162  img.dd
\end{center}
Unser Ergebniss mit \texttt{md5sum} betrug
\begin{center}
06d111e7ad654c1b7d47676fb6661540  image.dd
\end{center}
Mit \texttt{fdisk -l image.dd} entnahmen wir folgende Partitionsinformationen:
\lstset{basicstyle=\footnotesize}
\lstinputlisting{snippets/fdisk.bash}
Um die einzelnen Partitionen als Loopback Device mounten zu können muss ein Offset dem \texttt{losetup} Program übergeben werden. Dieser berechnet sich aus dem Startsektor sowie der Größe eines Sektors.\[startsector*sectorsize=offset\]
Aus dieser Berechnung ergeben sich folgende \texttt{losetup}-Befehle:

\begin{verbatim}
losetup -v -r -o      32256 /dev/loop0 image.dd
losetup -v -r -o 1801336320 /dev/loop1 image.dd
losetup -v -r -o 2204407296 /dev/loop2 image.dd
\end{verbatim}
\noindent Der Schalter \texttt{-r} verhindert, dass auf das Geraet geschrieben wird.

Um die Loopback Devices nun einzubinden musste mit \texttt{fsstat -t <Device>} das Dateisystem ermittelt werden. Angewendet auf die Geräte \texttt{/dev/loop0} und \texttt{/dev/loop2} erhielten wir \texttt{ext3}. Das Program gab, angewendet auf das Gerät \texttt{/dev/loop1}, \texttt{Cannot determine file system type} zurück. Durch die vorherige Ausgabe von \texttt{fdisk} war aber bereits der Typ \texttt{swap} gegeben.

Schließlich konnten die beiden Partitionen mit dem \texttt{ext3} Dateisystem mit \texttt{mount -t ext3 <Device>} eingebunden werden.