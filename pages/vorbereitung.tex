\section{Vorbereitung}
%Nach dem entpacken des Images stellten wir die in der Vorlesung erwähnte Abweichung der MD5 Summe fest.

%Die erwartete Summe war
%\begin{center}
%4afc088a94dd6c36e750b7462e737162  img.dd
%\end{center}
%Unser Ergebniss mit \texttt{md5sum} betrug
%\begin{center}
%06d111e7ad654c1b7d47676fb6661540  image.dd
%\end{center}
Nach dem entpacken des Images wurden mit \texttt{fdisk~-l~image.dd} folgende Partitionsinformationen entnommen:
\lstset{basicstyle=\footnotesize}
\lstinputlisting{snippets/fdisk.bash} %Bin noch nicht sicher ob ich das einbinden soll oder statisch hier rein schrieben MSK: Ja, warum nicht?
Da die Partitionstabelle des Images defekt ist (Startsektor = 0, normalerweise 63 bei der ersten Partition) und um die einzelnen Partitionen als Loopback Device mounten zu können muss dem \texttt{losetup} Programm ein Offset übergeben werden. Dieser berechnet sich aus dem Startsektor sowie der Größe eines Sektors.\[startsector*sectorsize=offset\]
Aus dieser Berechnung ergeben sich folgende \texttt{losetup}-Befehle:

\begin{verbatim}
losetup -v -r -o      32256 /dev/loop0 image.dd
losetup -v -r -o 1801336320 /dev/loop1 image.dd
losetup -v -r -o 2204407296 /dev/loop2 image.dd
\end{verbatim}
\noindent Der Schalter \texttt{-r} verhindert, dass auf das Geraet schreibend zugegriffen wird.\\

Um die Loopback Devices in das Dateisystem einzubinden muss mit \texttt{fsstat~-t~<Device>} das Dateisystem ermittelt werden. Angewendet auf die Geräte \texttt{/dev/loop0} und \texttt{/dev/loop2} gab das Programm den Typ \texttt{ext3} an. Das Programm gab weiterhin, angewendet auf das Gerät \texttt{/dev/loop1}, \texttt{Cannot determine file system type} zurück. Durch die vorherige Ausgabe von \texttt{fdisk} war aber bereits der Typ \texttt{swap} identifiziert.\\

Schließlich konnten die beiden Partitionen mit dem \texttt{ext3} Dateisystem mit \texttt{mount~-o ~ro~-t~ext3~<Device>~<Mountpoint>} eingebunden werden.