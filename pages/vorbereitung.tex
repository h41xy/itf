\section{Vorbereitung}
Nach dem entpacken des Images stellten wir die in der Vorlesung erwähnte Abweichung der MD5 Summe fest.

Die erwartete Summe war
\begin{center}
4afc088a94dd6c36e750b7462e737162  img.dd
\end{center}
Unser Ergebniss mit \texttt{md5sum} betrug
\begin{center}
06d111e7ad654c1b7d47676fb6661540  image.dd
\end{center}
Mit \texttt{fdisk -l image.dd} entnahmen wir folgende Partitionsinformationen:\\

\noindent
\texttt{
Disk image.dd: 3221 MB, 3221225472 bytes\\
255 heads, 63 sectors/track, 391 cylinders, total 6291456 sectors\\
Units = sectors of 1 * 512 = 512 bytes\\
Sector size (logical/physical): 512 bytes / 512 bytes\\
I/O size (minimum/optimal): 512 bytes / 512 bytes\\
Disk identifier: 0x0009d1b1\\
\\Device Boot      Start         End      Blocks   Id  System\\
image.dd1   *           0     3518171     1759086   83  Linux\\
image.dd2         3518235     4305419      393592+  82  Linux swap / Solaris\\
image.dd3         4305420     6281414      987997+   5  Extended\\
image.dd5         4305483     6281414      987966   83  Linux\\}

\lstset{basicstyle=\footnotesize}
\lstinputlisting{snippets/fdisk.bash}


Um die einzelnen Partitionen als Loopback Device mounten zu können muss ein Offset dem \texttt{losetup} Program übergeben werden. Dieser berechnet sich aus dem Startsektor sowie der Größe eines Sektors.\[startsector*sectorsize=offset\]
Aus dieser Berechnung ergeben sich folgende \texttt{losetup}-Befehle:
\begin{center}
\texttt{losetup -v -r -o 32256 /dev/loop0 image.dd}\footnote{Da sich auf der ersten Partition der MBR befindet muss als Startsektor 63 gewaehlt werden.}\\
\texttt{losetup -v -r -o 1801336320 /dev/loop1 image.dd}\\
\texttt{losetup -v -r -o 2204407296 /dev/loop2 image.dd}\\
\end{center}
%Entweder ich machich aus dem hier oben noch ne Tabelle oder irgendwie linksbeundig

Um die Loopback Devices nun einzubinden musste mit \texttt{fsstat} das Dateisystem ermittelt werden.

