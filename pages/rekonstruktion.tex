\section{Rekonstruktion des Vorfalls}
Die ersten fehlgeschlagenen Zugriffsversuche unternahm der Angreifer um \hl{referenz} 22:13:47 Uhr von der IP-Adresse 10.0.0.6 über ssh.
Anschließend wurden erfolgreiche Zugriffe auf den Webserver von der selben IP-Adresse im Access Log des Apache2 \hl{referenz} gefunden.
Der Angreifer hat dann die \texttt{upload.html} und \texttt{upload.php} gefunden und diese genutzt um die Dateien \texttt{config.php, phpshell.php} sowie \texttt{style.css} hochzuladen. Der Ordner \texttt{upload} in welchem diese Dateien gespeichert wurden besitzt die Rechte \texttt{read, write} und \texttt{execute} für jeden Benutzer \hl{referenz}. 
Dieser Umstand sowie der deaktivierte \texttt{safe\_mode} in \texttt{/etc/php5/apache2/php.ini} versetzten den Agreifer in die Lage eine PHP-Remote-Shell auszuführen.
Diese Shell erhielt die Berrechtigungen des \texttt{www-data} Benutzers, also die des Webservers, welcher die PHP-Shell ausführt.
Bei seiner Analyse des Systems fand der Angreifer in \texttt{/usr/local/bin/} die Binärdatei \texttt{mysudo} welche von Benutzer \texttt{itf} erstellt wurde und anschließend von \texttt{root} um 21:00 Uhr \hl{referenz auf bash history und timeline} das \texttt{setuid}-Bit gesetzt bekommen hat sowie den Eigentümerwechsel auf \texttt{root:root}.
Unter Verwendung der \texttt{phpshell} und \texttt{mysudo} wurden dann die Benutzer \texttt{evil} (um 22:50:04 Uhr), \texttt{evil2} (um 22:56:35 Uhr) und \texttt{evil3} (um 23:00:10 Uhr) angelegt. Mit dem Benutzer \texttt{evil} loggte der Angreifer sich um 23:23:02 Uhr von der IP-Adresse 10.0.0.66 per \texttt{ssh} ein. Diese SSH-Session ging bis um 23:29:11 Uhr. In dieser Zeit wurde ein Backdoor eingebaut. Um dies zu realsieren wurde die Datei \texttt{/etc/init.d/networking} \hl{referenz} angepasst und die Datei \texttt{netcat} um 23:08:45 Uhr hochgeladen Somit konnte das System von außen, auch bei deaktiviertem SSH-Server und entfernten Usern, mit Rootrechten gesteuert werden.

Somit war der Server nun unter Kontrolle des Angreifers und konnte auch nach einer oberflächlichen Bereinigung durch den Administrator, sprich Entfernung der User und Beheben der Sicherheitslücken des Webservers, weiterhin mit allen Berechtigungen aus der Ferne gesteuert werden. Erst eine Bereinigung der \texttt{/etc/init.d/networking} und eine Inbetriebnahme einer restriktiven Firewall hätten das Problem beseitigt.