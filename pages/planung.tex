\section{Planung}
Um den Vorfall zu bearbeiten und die gestellte Aufgabe zu lösen sammelten wir in einem Brainstorming Ideen was alles überprüft werden sollte. Um den Projektablauf besser zu strukturieren kategorisierten wir diese anschließend und formulierten sie als Tests welche durchgefuehrt werden sollten. Wir gestalteten die Tests als Einstiegspunkte in unsere Untersuchung da wir erwarteten je nach Ergebniss unsere Aktionen sehr dynamisch fortführen zu müssen.

\begin{itemize}
\item Timeline\\
Um einen Überblick der Aktionnen auf dem Dateisystem zu erhalten sollte eine Timeline erstellt werden.

\item Logins\\
Die Datei \texttt{/var/log/auth.log} soll auf Auffaelligkeiten überprüft werden. Sie enthält alle login und logout Ergeignisse von normalen Benutzern sowie System Prozessen. Wir erhoffen uns aus diesen Ereignissen weiterführende Hinweise auf nicht authorisierte Zugriffe auf das System.

\item Benutzer\\
In der Datei \texttt{/etc/shadow} sind alle Benutzer des Systems hinterlegt. Sollten Benutzer hier hinterlegt sein welche Auffaelligkeiten aufweisen wie Name oder Erstellungsdatum können wir verfolgen was der Benutzer am System verändert hat.

\item Module\\
Die Datei \texttt{/proc/modules} listet alle geladenen Module des Systems. Sollten auffaellige Module geladen sein kann mittels Reverse Engineering deren Zweck ermittelt werden.

\item Swap space\\
In dieser Untersuchung wird der swap space nicht berücksichtigt.
\end{itemize}